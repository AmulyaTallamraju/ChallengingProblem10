\documentclass[journal,12pt,twocolumn]{IEEEtran}

\usepackage{setspace}
\usepackage{gensymb}
\singlespacing
\usepackage[cmex10]{amsmath}

\usepackage{amsthm}

\usepackage{mathrsfs}
\usepackage{txfonts}
\usepackage{stfloats}
\usepackage{bm}
\usepackage{cite}
\usepackage{cases}
\usepackage{subfig}

\usepackage{longtable}
\usepackage{multirow}

\usepackage{enumitem}
\usepackage{mathtools}
\usepackage{steinmetz}
\usepackage{tikz}
\usepackage{circuitikz}
\usepackage{verbatim}
\usepackage{tfrupee}
\usepackage[breaklinks=true]{hyperref}
\usepackage{graphicx}
\usepackage{tkz-euclide}

\usetikzlibrary{calc,math}
\usepackage{listings}
    \usepackage{color}                                            %%
    \usepackage{array}                                            %%
    \usepackage{longtable}                                        %%
    \usepackage{calc}                                             %%
    \usepackage{multirow}                                         %%
    \usepackage{hhline}                                           %%
    \usepackage{ifthen}                                           %%
    \usepackage{lscape}     
\usepackage{multicol}
\usepackage{chngcntr}

\DeclareMathOperator*{\Res}{Res}

\renewcommand\thesection{\arabic{section}}
\renewcommand\thesubsection{\thesection.\arabic{subsection}}
\renewcommand\thesubsubsection{\thesubsection.\arabic{subsubsection}}

\renewcommand\thesectiondis{\arabic{section}}
\renewcommand\thesubsectiondis{\thesectiondis.\arabic{subsection}}
\renewcommand\thesubsubsectiondis{\thesubsectiondis.\arabic{subsubsection}}


\hyphenation{op-tical net-works semi-conduc-tor}
\def\inputGnumericTable{}                                 %%

\lstset{
%language=C,
frame=single, 
breaklines=true,
columns=fullflexible
}
\begin{document}

\newcommand{\BEQA}{\begin{eqnarray}}
\newcommand{\EEQA}{\end{eqnarray}}
\newcommand{\define}{\stackrel{\triangle}{=}}
\bibliographystyle{IEEEtran}
\raggedbottom
\setlength{\parindent}{0pt}
\providecommand{\mbf}{\mathbf}
\providecommand{\pr}[1]{\ensuremath{\Pr\left(#1\right)}}
\providecommand{\qfunc}[1]{\ensuremath{Q\left(#1\right)}}
\providecommand{\sbrak}[1]{\ensuremath{{}\left[#1\right]}}
\providecommand{\lsbrak}[1]{\ensuremath{{}\left[#1\right.}}
\providecommand{\rsbrak}[1]{\ensuremath{{}\left.#1\right]}}
\providecommand{\brak}[1]{\ensuremath{\left(#1\right)}}
\providecommand{\lbrak}[1]{\ensuremath{\left(#1\right.}}
\providecommand{\rbrak}[1]{\ensuremath{\left.#1\right)}}
\providecommand{\cbrak}[1]{\ensuremath{\left\{#1\right\}}}
\providecommand{\lcbrak}[1]{\ensuremath{\left\{#1\right.}}
\providecommand{\rcbrak}[1]{\ensuremath{\left.#1\right\}}}
\theoremstyle{remark}
\newtheorem{rem}{Remark}
\newcommand{\sgn}{\mathop{\mathrm{sgn}}}
\providecommand{\abs}[1]{\vert#1\vert}
\providecommand{\res}[1]{\Res\displaylimits_{#1}} 
\providecommand{\norm}[1]{\lVert#1\rVert}
%\providecommand{\norm}[1]{\lVert#1\rVert}
\providecommand{\mtx}[1]{\mathbf{#1}}
\providecommand{\mean}[1]{E[ #1 ]}
\providecommand{\fourier}{\overset{\mathcal{F}}{ \rightleftharpoons}}
%\providecommand{\hilbert}{\overset{\mathcal{H}}{ \rightleftharpoons}}
\providecommand{\system}{\overset{\mathcal{H}}{ \longleftrightarrow}}
	%\newcommand{\solution}[2]{\textbf{Solution:}{#1}}
\newcommand{\solution}{\noindent \textbf{Solution: }}
\newcommand{\cosec}{\,\text{cosec}\,}
\providecommand{\dec}[2]{\ensuremath{\overset{#1}{\underset{#2}{\gtrless}}}}
\newcommand{\myvec}[1]{\ensuremath{\begin{pmatrix}#1\end{pmatrix}}}
\newcommand{\mydet}[1]{\ensuremath{\begin{vmatrix}#1\end{vmatrix}}}
\numberwithin{equation}{subsection}
\makeatletter
\@addtoreset{figure}{problem}
\makeatother
\let\StandardTheFigure\thefigure
\let\vec\mathbf
\renewcommand{\thefigure}{\theproblem}
\def\putbox#1#2#3{\makebox[0in][l]{\makebox[#1][l]{}\raisebox{\baselineskip}[0in][0in]{\raisebox{#2}[0in][0in]{#3}}}}
     \def\rightbox#1{\makebox[0in][r]{#1}}
     \def\centbox#1{\makebox[0in]{#1}}
     \def\topbox#1{\raisebox{-\baselineskip}[0in][0in]{#1}}
     \def\midbox#1{\raisebox{-0.5\baselineskip}[0in][0in]{#1}}
\vspace{3cm}
\title{Challenging Problem 10}%number
\author{Amulya Tallamraju - AI20BTECH11003}
\maketitle
\newpage
\bigskip
\renewcommand{\thefigure}{\theenumi}
\renewcommand{\thetable}{\theenumi}
\newcommand*{\permcomb}[4][0mu]{{{}^{#3}\mkern#1#2_{#4}}}
\newcommand*{\perm}[1][-3mu]{\permcomb[#1]{P}}
\newcommand*{\comb}[1][-1mu]{\permcomb[#1]{C}}
Download all python codes from 
\begin{lstlisting}
https://github.com/AmulyaTallamraju/ChallengingProblem10/blob/main/ChallengingProblem10/codes/ChallengingProblem10.py
\end{lstlisting}
%
and latex-tikz codes from 
%
\begin{lstlisting}
https://github.com/AmulyaTallamraju/ChallengingProblem10/blob/main/ChallengingProblem10/ChallengingProblem10.tex
\end{lstlisting}
\section*{UGC DEC 2018 MATH SET a Q 114}
Suppose that $X_1,X_2,X_3,...,X_{10}$ are i.i.d, N(0,1). Which of the following statements are correct ?
\begin{enumerate}[label = (\Alph*)]
    \item $\pr{X_1>X_2+X_3+...+X_{10}}=\frac{1}{2}$
    \item $\pr{X_1>X_2X_3...X_{10}}=\frac{1}{2}$
    \item $\pr{\sin{X_1}>\sin{X_2}+\sin{X_3}+...+\sin{X_{10}}}=\frac{1}{2}$
    \item $\pr{\sin{X_1}>\sin({X_2+X_3+\ldots+X_{10}})}=\frac{1}{2}$
\end{enumerate}
\section*{Solution}
The random variables follow the standard normal distribution. If $X \sim \mathcal{N}(0,1)$ then $Y =-X$ also follows standard normal distribution. This can be proved in the following way:
\begin{align}
    P(Y \leq u) &= P(-X \leq u) \\
    &= P(X > -u) \\
    &= 1 - P(X \leq -u) \\
    &= 1 - (1 - P(X \leq u) \\
    &= P(X \leq u) 
\end{align}
As the distribution is symmetric, \begin{align}
 P(X\leq -u)=P(X\geq u)= 1-P(X\leq u)   
\end{align} 
Therefore, $-X_1,-X_2 ... -X_{10}$ follow the same distribution as $X_1,X_2...X_{10}$

If 
\begin{align}
 X_1,X_2...X_n \sim \mathcal{N}(0,1)   
\end{align}
where n is an even number and g(x) is an odd function, then
\begin{multline}
    \pr{g(X_1)>\sum_{k=2}^ng(X_k)}\\=
    \pr{g(-X_1)<\sum_{k=2}^ng(-X_k)}\\=
    \pr{g(X_1)<\sum_{k=2}^ng(X_k)}
\end{multline}
As they follow the same distribution, the above expression is true. Thus we have
\begin{align}\label{equality}
    \pr{g(X_1)>\sum_{k=2}^ng(X_k)}=
    \pr{g(X_1)<\sum_{k=2}^ng(X_k)}
\end{align}
Also, as $X_1$ is a continuous random variable
\begin{align}
    \pr{g(X_1)=\sum_{k=2}^ng(X_k)}=0
\end{align}
As the cases
\begin{align}
    g(X_1)>\sum_{k=2}^ng(X_k)
\end{align}
and
\begin{align}
    {g(X_1)<\sum_{k=2}^ng(X_k)}
\end{align}
are complementary to each other and from \eqref{equality} we have
\begin{align}\label{sum}
 \pr{g(X_1)>\sum_{k=2}^ng(X_k)}=\frac{1}{2}    
\end{align}


\begin{multline}
    \pr{g(X_1)>\prod_{k=2}^ng(X_k)}\\=
    \pr{g(-X_1)<\prod_{k=2}^ng(-X_k)}\\=
    \pr{g(X_1)<\prod_{k=2}^ng(X_k)}
\end{multline}
As they follow the same distribution, the above expression is true. Thus we have
\begin{align}\label{equality}
    \pr{g(X_1)>\prod_{k=2}^ng(X_k)}=
    \pr{g(X_1)<\prod_{k=2}^ng(X_k)}
\end{align}
Also, as $X_1$ is a continuous random variable
\begin{align}
    \pr{g(X_1)=\prod_{k=2}^ng(X_k)}=0
\end{align}
As the cases
\begin{align}
    g(X_1)>\prod_{k=2}^ng(X_k)
\end{align}
and
\begin{align}
    {g(X_1)<\prod_{k=2}^ng(X_k)}
\end{align}
are complementary to each other and from 
 \eqref{equality} we have
\begin{align}\label{prod}
 \pr{g(X_1)>\prod_{k=2}^ng(X_k)}=\frac{1}{2}    
\end{align}
\begin{enumerate}[label = (\Alph*)]
    \item 
    From \eqref{sum} , taking $g(x)=x$,
    \begin{align}
        \pr{X_1>X_2+...+X_{10}}=\frac{1}{2}
    \end{align}
\item
From \eqref{prod} taking $g(x)=x$
        \begin{align}
         \pr{X_1>X_2X_3...X_{10}}=\frac{1}{2}   
        \end{align}
\item 
From \eqref{sum} taking $g(x)=\sin{x}$
        \begin{align}
         \pr{\sin{X_1}>\sin{X_2}+...+\sin{X_{10}}}=\frac{1}{2}   
        \end{align}
\item \begin{multline}
        \pr{\sin{X_1}>\sin{(X_2+...+X_{10})}}\\=\pr{\sin{(-X_1)}<\sin{(-X_2-...-X_{10})}}\\
        =\pr{\sin{X_1}<\sin{(X_2+...+X_{10})}}
    \end{multline}
    As they follow the same distribution, the above expression is true.
    Thus we have
    \begin{multline}\label{2} 
    \pr{\sin{X_1}>\sin{(X_2+...+X_{10}}))}\\
        =\pr{\sin{X_1}<\sin{(X_2+...+X_{10}})}    
    \end{multline}
    Also, as $X_1$ is a continuous random variable
    \begin{align}
       \pr{\sin{X_1}=\sin{(X_2+...+X_{10})}}=0
    \end{align}
     As the cases
     \begin{align}
      {\sin{X_1}>\sin{(X_2+...+X_{10})}}   
     \end{align}and 
     \begin{align}
         {\sin{X_1}<\sin{(X_2+...+X_{10})}} 
     \end{align}are complementary to each other and from \eqref{2}  , we have 
        \begin{align}
         \pr{\sin{X_1}>\sin{(X_2+...+X_{10}})}=\frac{1}{2}   
        \end{align}

\end{enumerate}
\end{document}